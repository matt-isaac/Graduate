\documentclass[handout, xcolor=dvipsnames]{beamer}


\mode<presentation> {

\usetheme{Berlin}
\usetheme{Warsaw}

\setbeamertemplate{itemize items}[ball] % if you want a ball
\setbeamertemplate{itemize subitem}[circle] % if you wnat a circle
\setbeamertemplate{itemize subsubitem}[triangle] % if you want a triangle

\setbeamertemplate{blocks}[rounded][shadow=true]
\setbeamertemplate{navigation symbols}{}
\setbeamertemplate{mini frames}[square]
\setbeamersize{text margin left=6mm, text margin right=4mm}
\setbeamercolor{button}{bg = white, fg = blue}


\defbeamertemplate*{footline}{my theme}
 {
  \leavevmode%
  \hbox{%
  \begin{beamercolorbox}[wd=.35\paperwidth,ht=2.25ex,dp=1ex,center]{author in head/foot}%
    \usebeamerfont{author in head/foot}\insertshortauthor%~~(\insertshortinstitute USU)
  \end{beamercolorbox}%
  \begin{beamercolorbox}[wd=.65\paperwidth,ht=2.25ex,dp=1ex,center]{title in head/foot}%
    \usebeamerfont{title in head/foot}\insertshorttitle
  \end{beamercolorbox}%
  \begin{beamercolorbox}[wd=.1\paperwidth,ht=2.25ex,dp=1ex,right]{date in head/foot}%
     %\usebeamerfont{date in head/foot}\insertshortdate{}\hspace*{2em}
    %\insertframenumber{}\hspace*{2ex} 
  \end{beamercolorbox}}%
  \vskip0pt%
}

 % or ...

  \setbeamercovered{transparent}
  % or whatever (possibly just delete it)
}


\usetheme{Darmstadt}
\usefonttheme[onlylarge]{structurebold}
\setbeamerfont*{frametitle}{size=\normalsize,series=\bfseries}
\setbeamertemplate{navigation symbols}{}

\usepackage[english]{babel}
\usepackage{multirow}
\usepackage[latin1]{inputenc}
\usepackage{times}
\usepackage[T1]{fontenc}
\usepackage{graphicx}
\usepackage{natbib} %citep and citet
\usepackage{color}
\usepackage{amsmath}

%%% for references
\renewcommand{\bibsection}{\subsubsection*{\bibname}} % to avoid References creates new section/subsection in header
\bibpunct[:]{(}{)}{;}{a}{}{,}




\begin{document}

% here you define the information that will be displayed in the title/cover page

\title[\hspace*{0.5cm} Statistical Visualization II, Utah State University
	   \hspace*{80pt}\hfill \insertframenumber\hspace*{0.5cm}]
       {An Interactive Tool to Visualize Results from Uncertainty Quantification}


\author[\hspace*{0cm}Matt Isaac --- April 9, 2019\hspace*{0.25cm}]
		{Matt Isaac\\
		e-mail: \tt{matt.isaac@aggiemail.usu.edu} \\[-0.8cm]
		~ \\
		% \flushright\includegraphics[height=2.0cm]{figures/Baltimore_P5260032}
		}


\date{April 9, 2019}


% here you build the title page
\frame{
  \titlepage
}


% outline 
\AtBeginSection[]
{
  \begin{frame}<beamer>{outline}
    \frametitle{Outline}
    \small
    \tableofcontents[currentsection]%,hideothersubsections]
    \normalsize
  \end{frame}
}


% If you wish to uncover everything in a step-wise fashion, uncomment
% the following command:

%\beamerdefaultoverlayspecification{<+->}


%%%%%%%%%%%%%%%%%%%%%%%%%
%%%   Outline           %
%%%%%%%%%%%%%%%%%%%%%%%%%

\begin{frame} 
\frametitle{Outline}
  \tableofcontents
  % You might wish to add the option [pausesections]
\end{frame}

\newcounter{finalframe}

%%%%%%%%%%%%%%%%%%%%%%%%%
\section{Introduction}  
%%%%%%%%%%%%%%%%%%%%%%%%%

\subsection{}
\begin{frame}
	\frametitle{Motivation}
	Uncertainty Quantification (UQ):
	\begin{itemize}
    	\item Is a branch of simulation analysis that is used often in engineering contexts; 
    	\item Assists engineers and decision-makers in finding a balance between design efficiency and design sufficiency (avoid under/over designing);
    	\item Provides a quantification of how variability in a system can influence the end state of that system. 
    \end{itemize}
\end{frame}


\subsection{}
\begin{frame}
	\frametitle{UQ Overview - Terminology}
	\begin{description}
	  \item[system response quantity (SRQ):] the output (i.e. prediction) from an engineering model.
		  \item[engineering model:] A mathematical model that defines the relationship between the parameters (model inputs) and SRQ (model output). 
		\item[aleatory uncertainty:] Uncertainty from randomness inherent to a given parameter. This type of uncertainty is irreducible.
	  \item[epistemic uncertainty:] Uncertainty from a lack of knowledge about a given parameter. This type of uncertainty can be reduced by more information about the parameter.


	\end{description}
\end{frame}


\begin{frame}
	\frametitle{UQ Overview - Algorithm}
	Example: A beam required to bear a certain load. The SRQ is a representation of the capability of the beam.
	
	\[\text{Aleatory model inputs: }A_1, A_2, A_3 \]
	\[\text{Epistemic model inputs: }E_1, E_2, E_3\]
	\[ SRQ = E_1^{E_2}A_3A_2^{A_1} + E_3\]
	\\
	\begin{itemize}
	\item A domain expert selects a probability distribution for each input.
	\end{itemize}	
	\begin{center}
	{\scriptsize Example adapted from \cite{EW2018}}
	\end{center}
\end{frame}

\begin{frame}
	\frametitle{UQ Overview - Algorithm}
	\begin{center}
	\begin{center} 
	\includegraphics[height=6.5cm,width=8.25cm]{figures/uq_algo.png}
	\end{center}	
	{\tiny Diagram adapted from \cite{EW2018}}
	\end{center}
\end{frame}

\begin{frame}
	\frametitle{Project Goal}
	\textbf{Goal:} Develop an interactive tool to create meaningful visualizations (like the one below) and useful interpretations of results from a UQ simulation.\\
	\begin{center}
	\begin{center} 
	\includegraphics[height=4cm,width=5cm]{figures/viz_exmp.png}
	\end{center}	
	\end{center}
\end{frame}

%%%%%%%%%%%%%%%%%%%%%%%%%
\section{Methods}  
%%%%%%%%%%%%%%%%%%%%%%%%%

\subsection{}
\begin{frame}
	\frametitle{Key Features}
		\begin{itemize}
		\setlength\itemsep{1.5em}
		\item upload data via a {\tt .csv} file
		\item toggle P-box and CDF's on/off
		\item select upper/lower percentiles to be used in P-box calculation
		\item adjustable transparency for CDF ensemble
		\item extract either a probability interval or an SRQ interval from P-box
		\item download and save visualization as a {\tt .png} file
    \end{itemize}
\end{frame}

\begin{frame}
	\frametitle{Planned Layout}
	\begin{center}
	\includegraphics[height=7cm,width=11cm]{figures/proto_app_layout.png}
	\end{center}	
\end{frame}

\subsection{}
\begin{frame}
	\frametitle{{\tt R} Packages}
	\begin{description}[labelwidth=4em]

\item[{\bf shiny:}] {\footnotesize The {\tt shiny} package \citep{SHINY} provided the framework on which the application was developed and deployed. It also contains the implementations for all user-interface components (toggle buttons, check boxes, numeric inputs, sliders, etc.).}

\item[{\bf ggplot2:}] {\footnotesize The plotting functionality of the {\tt ggplot2} package \citep{GGPLOT} was used to generate the actual visualization and to add, remove, or adjust components on the plot.}

\item[{\bf shinydashboard:}] {\footnotesize The {\tt shinydashboard} package \citep{DASH} was used as a wrapper around the {\tt shiny} framework to give a clean, polished appearance to the application.}

\item[{\bf shinyalert:}] {\footnotesize The {\tt shinyalert} package \citep{SALERT} was used to generate pop-up messages.}

\item[{\bf shinyWidgets:}] {\footnotesize The {\tt shinyWidgets} package \citep{SWIDGET} was used to to create custom toggle buttons.}

\end{description}
\end{frame}


%%%%%%%%%%%%%%%%%%%%%%%%%
\section{Results}  
%%%%%%%%%%%%%%%%%%%%%%%%%

\subsection{}
\begin{frame}
	\frametitle{Deployment}
	\begin{center} 
		Shiny App currently deployed at: \\ {\tt https://misaac.shinyapps.io/UQViz/}
	\end{center}
\end{frame}

\subsection{}
\begin{frame}
	\frametitle{Data Source}
	\begin{itemize}
	\item Users can upload {\tt .csv} files for visualization. The files should have the x values in the 1st column, and the CDFs in subsequent columns. 
	\item For convenience, a sample data set is also provided
	\end{itemize}
	\begin{center} 
		\includegraphics[height=3cm,width=7cm]{figures/tab_ds.png}
	\end{center}
\end{frame}

\subsection{}
\begin{frame}
	\frametitle{Visualization Options}
	\begin{itemize}
	\item Several options are provided to allow customized visualizations 
	\end{itemize}
	\begin{center} 
		\includegraphics[height=2.25cm,width=10cm]{figures/tab_vo.png}
	\end{center}
\end{frame}

\subsection{}
\begin{frame}
	\frametitle{Visualization Options \& Examples}
	\begin{center} 
		\includegraphics[height=2.25cm,width=11cm]{figures/tab_vo.png}
	\end{center}
	\begin{figure}
	  \includegraphics[height=3cm,width=3.5cm]{figures/pbx.png}
	  \includegraphics[height=3cm,width=3.5cm]{figures/pbx_cdf_l.png}
	  \includegraphics[height=3cm,width=3.5cm]{figures/pbx_cdf_d.png}
	\end{figure}
\end{frame}

\subsection{}
\begin{frame}
	\frametitle{Labeling}
	\begin{itemize}
	\item Custom title and x-axis label can be added to adapt visualization to domain. 
	\end{itemize}
	\begin{center} 
		\includegraphics[height=1.5cm,width=7cm]{figures/tab_lb.png} \\
		\includegraphics[height=4cm, width=6cm]{figures/plt_labeled.png}
	\end{center}
\end{frame}

\subsection{}
\begin{frame}
	\frametitle{Interpretation - Extract Probability or SRQ Intervals}
  There are two ways a P-box can be used for interpretation:
	\begin{itemize}
	  \item{Input SRQ value to extract probability interval}
	  \item{Input probability value to extract SRQ interval}
	\end{itemize}
\end{frame}

\subsection{}
\begin{frame}
  \frametitle{Interpretation - Extract Probability Interval}
  \begin{center}
    \includegraphics[height=1.25cm, width=3.25cm]{figures/prob_int_ctrl.png} \\
    \includegraphics[height=5.75cm, width=8cm]{figures/prob_int.png}
  \end{center}
\end{frame}

\subsection{}
\begin{frame}
  \frametitle{Interpretation - Extract SRQ Interval}
  \begin{center}
    \includegraphics[height=1.25cm, width=3.25cm]{figures/srq_int_ctrl.png} \\
    \includegraphics[height=5.75cm, width=8.25cm]{figures/srq_int.png}
  \end{center}
\end{frame}

%%%%%%%%%%%%%%%%%%%%%%%%%
\section{Discussion}  
%%%%%%%%%%%%%%%%%%%%%%%%%

\subsection{}
\begin{frame}
  \frametitle{Challenges}
  \begin{itemize}
    \item{Long render time for hundereds (or thousands!) of traces on a plot}
    \item{Reasonable workaround: If the user wishes to display the CDFs on the final version of the plot, adjust all visualization options and controls while the CDFs are hidden, keeping the plot rendering times to a reasonable length. Then, as the last step, show the CDF ensemble.}
  \end{itemize}
\end{frame}

\subsection{}
\begin{frame}
  \frametitle{Future Work}
    \begin{itemize}
      \item{Research and implement other possible ways to visualize UQ results}
      \item{Create a self contained UQ simulation tool -- perform entire process, start to finish}
      \begin{itemize}
        \item{Define a mathematical model}
        \item{Select parameters as aleatory or epistemic}
        \item{Specify the number of inner and outer loop iterations}
        \item{Choose visualization type}
        \item{Interact with and customize dynamic visualization}
      \end{itemize}
    \end{itemize}
\end{frame}
      
\subsection{}
\begin{frame}
	\frametitle{}
	\begin{center}
  {\large \bf Questions ??} --- \\[1cm]
    or e-mail: \tt{matt.isaac@aggiemail.usu.edu}
  \end{center}
\end{frame}

\subsection{}
\begin{frame}
	\frametitle{Sources}

	\bibliographystyle{elsarticle-harv}
	
  {\footnotesize
	\bibliography{references2}}


\end{frame}


\end{document}

